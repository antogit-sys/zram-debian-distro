%definisce tipo di documento
\documentclass{article}
\usepackage{tcolorbox}
\usepackage{graphicx}

\title{Cos'è la zram e come usarla su sistemi Debian}
\author{Antonio Perrucci (Mr. Glitch)} 

\begin{document}
	\maketitle
	\hrule
	\newpage

	\section{Avere pochissima ram}
		\paragraph{Pochissima RAM? Usiamo zRam.} Se abbiamo pochissima  ram risulta  
		difficile e nello stesso tempo fastidioso aprire più programmi nello stesso tempo
		perché il nostro computer \textbf{andrà lentissimo} a causa  di sovraccarico della memoria
		ram e a causa di un sovraccarico della memoria di swap(se ne abbiamo una). 
		La soluzione per chi non ha possibilità di comprare una nuova ram o per chi è tirchio
		è solo una! usare la \textbf{ZRAM}  
	
	\section{Ma che cos'è  la zram su linux?}
		\paragraph{zram} la zram precedentemente chiamata \textbf{compcache} è un modulo presente
		nel kernel linux che va a migliorare le performance del nostro sistema comprimendo alcuni 
		blocchi di Ram

		\paragraph{Ma come funzion di preciso la  zram?}
		 Il suo funzionamento è complesso, ma riassumibile in poche righe:  \textit{i dati in RAM vengono compressi} e
		immagazzinati in una \textbf{partizione swap virtuale caricata in memoria RAM} che occupa fino al 25\% della
		stessa. Raggiunta la soglia critica i pacchetti vengono compressi al massimo e swappati dentro la zRam, con un 		enorme recupero di prestazioni e di reattività (la compressione/decompressione è circa 20 volte più veloce
		dell’accesso diretto alla swap sul disco rigido). 
		In teoria è come aggiungere un modulo di RAM in più della capienza dimezzata rispetto all’originale: su 1GB di
		RAM si ottengono in totale circa 1,50 GB. \textbf{Attenzione però! Il modulo
		zRAM funziona solo su macchine che hanno processori non scrausi}. Infatti, quando utilizziamo il modulo,
		in realtà stiamo sfruttando notevolmente il processore, che quindi dovrà essere impegnato nelle operazioni sia di
		compressione che di decompressione. Il modulo zRAM è considerato stabile a partire dal kernel 3.14. 

		\section{Passiamo al pratico} per usare la zram si linux ci sono molti modi, noi utilizzeremo il piu semplice e il
		automatico, ci bastera scrivere sul terminale
		\begin{tcolorbox}
                sudo apt install zram-config
		\end{tcolorbox}
            per avviarlo ad ogni avvio del sistema digitare il comando 
            \begin{tcolorbox}
               sudo systemctl enable zram-config.service 
		\end{tcolorbox}
            succettivamente per vedere lo stato del servizio digitare
            \begin{tcolorbox}
               systemctl status zram-config.service 
		\end{tcolorbox}

            \section{Automatizziamo un po il tutto}
            digitare sul terminale 
            \begin{tcolorbox}
             nano \$HOME/.bashrc
            \end{tcolorbox}
            è scriviamo alla fine del file:
            \begin{tcolorbox}
            alias zstatus="systemcyl status zram-config"
            \end{tcolorbox}

            \textbf{N.B non ci devono essere spazi tra in più}
            
            aprendo una nuova istanza del terminale e digitando lo pseudo-comando zstatus 
            apparirà nello schermo lo stato del daemon \textbf{zram-config.service} come mostrato in figura \ref{f1}
            
            \begin{figure}
                \centering
                \includegraphics[scale=1]{image}
                \caption{status del demone zram-config}
                \label{f1}
            \end{figure}       
            
\end{document}
